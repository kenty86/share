\documentclass[a4paper,11pt]{ltjsarticle}


% 数式
\usepackage{amsmath,amssymb,amsfonts}
\usepackage{mathtools}
\usepackage{bm}
% 画像
\usepackage[dvipdfmx]{graphicx}
%色
\usepackage{color}

\begin{document}

\title{線形システム解析\\後半レポート}
\author{2022531033 関川謙人}
\date{\today}
\maketitle
\section*{問1}
差分方程式
\begin{equation*}
  y_{k+1} = 0.5y_{k}+0.5u_{k}
\end{equation*}
のインパルス応答を求めよ
\begin{gather*}
  y_{k} = CA^{k}x_{0}+CA^{k-1}Bu_{0}+CABu_{k-2}+CBu_{k-1}\dotsb \\
  h_{k}を、\\
  h_{k}=
  \begin{cases}
    CA^{k-1}B & (k \geq 1) \\
    0 & (k < 1)
  \end{cases}
  とおくと、\\
  y_{k}は、\\
  \begin{split}
    y_{k} &= h_{k}u_{0}+h_{k-1}u_{1}+\dotsb+h_{2}u_{k-2}+h_{1}u_{k-1}\\
    &= \sum_{i=0}^{k-1}h_{k-1}u_{i}
  \end{split}
\end{gather*}
この時の$h_{k}$を\textcolor{red}{インパルス応答}と呼ぶ。
\begin{gather*}
  この問いにおいて、y_{k}とx_{k}の関係式は\\
  \left(
    \begin{split}
    x_{k+1} &= 0.5x_{k} + 0.5u_{k} \\
    y_{k} &= x_{k}
    \end{split}
  \right.
  であり、\\
  \left(
    \begin{split}
      x_{k+1} &= Ax_{k} + Bu_{k} \\
      y_{k} &= Cx_{k}
    \end{split}
  \right.
  である。
  \\係数比較すると、
  \begin{pmatrix}
    A = 0.5 \\
    B = 0.5 \\
    C = 1 \\
  \end{pmatrix}
  \\
  \\ここで、インパルス入力は、
  \\u_{k}=
  \begin{cases}
    1 & (k = 0) \\
    0 & (k \neq 0)
  \end{cases}
  \\であるため、インパルス応答h_{k}は、
  \begin{cases}
    0.5^{k} & (k = 0) \\
    0 & (k \neq 0)
  \end{cases}
\end{gather*}
となる。
\section*{問2}
時系列$x_{k}$を、\\
\begin{equation*}
  x_{k} =
  \begin{cases}
    1 & (k \geqq 0かつkが偶数) \\
    0 & (k \geqq 0かつkが奇数) \\
    0 & (それ以外)
  \end{cases}
\end{equation*}
とする。この$x_{k}$のZ変換$X(z)$を求めよ。\\
時系列$\lbrace{x_{k}}\rbrace^{\infty}_{k=-\infty}$に対し、\\
\textcolor{red}{
\begin{equation}
  X_(z) = \sum_{i = -\infty}^{\infty}x_{i}z^{-i}
\end{equation}
}
\begin{gather*}
  \begin{split}
    X(z) &= Z[x_{k}] \\
    &= \sum_{i =- \infty}^{\infty}u_{i}z^{-i} = \sum_{i=0}^{\infty}u_{i}z^{-i}\\
    &= 1 - z^{-1} + z^{-2} + \dotsb -z^{-2k-1} + z^{-2k} \\
    &= \frac{1}{1-(-z^{-1})} \ (|x^{-1}| < 1) \\
    &= \frac{1}{1+z^{-1}} = \frac{z}{z+1}
  \end{split}
\end{gather*}
\section*{問3}
インパルス応答が\\
\begin{equation*}
  h_{k}=
  \begin{cases*}
    0.8^{k-1} & (k \geqq 1) \\
    0 & (k < 1)
  \end{cases*}
\end{equation*}
\\のシステムにステップ入力\\
\begin{equation*}
  u_{k}=
  \begin{cases*}
    1 & (k \geqq 0) \\
    0 & (k < 0)
  \end{cases*}
\end{equation*}
\\
を加えたときの応答$y_{k}$を求めよ。

$H(z),U(z)$を求め、$Y(z)=H(z)U(z)$を求めて逆Z変換することで
応答を求めることができる。

まず、$H(z),U(z)$を求める。\\
\begin{gather*}
  \begin{split}
    H(z) 
    &= Z[h_{k}] \\
    &= \sum_{i = -\infty}^{\infty}h_{k}z^{-i} \\
    &= \sum_{i = 1}^{\infty}0.8^{k-1}z^{-i} \\
    &= z^{-1} + 0.8z^{-2} + 0.8^{2}z^{-3} \dotsb \\
    &= \frac{z^{-1}}{1 - 0.8z^{-1}} \ (|z^{-1}| < 1) \\
    &= \frac{1}{z-0.8}
  \end{split}
\end{gather*}

\begin{gather*}
  \begin{split}
    U(z) 
    &= Z[u_{k}]\\
    &= \sum_{i = -\infty}^{\infty}u_{k}z^{-i} \\
    &= \sum_{i = 0}^{\infty}1z^{-i} \\
    &= 1 + z^{-1} + z^{-2} + \dotsb \\
    &= \frac{1}{1 - z^{-1}} \ (|z^{-1}| < 1) \\
    &= \frac{z}{z-1}
  \end{split}
\end{gather*}
$Y(z) = H(z)U(z)$であるため、\\
\begin{gather*}
  \begin{split}
    Y(z) 
    &=\frac{1}{z-0.8} ・ \frac{z}{z-1} \\
    &=\frac{z}{(z-0.8)(z-1)}
  \end{split}
\end{gather*}
\\部分分数分解を用いて、
\begin{equation*}
  Y(z) = \frac{a}{(z-0.8)} + \frac{b}{(z-1)}
\end{equation*}
\\このとき両辺に$(z-0.8)$をかけると、
\begin{equation*}
  (z-0.8)Y(z) = a + \frac{(z-0.8)}{(z-1)}b
\end{equation*}
この時、$z → 0.8$において、
\begin{equation*}
  (z-0.8)Y(z) = a + 0
\end{equation*}
である。よって、
\begin{gather*}
  \begin{split}
    a &= \lim_{z → 0.8}(z-0.8)Y(z) \\
    a &= \frac{z}{z-1} \\
      &= \frac{0.8}{-0.2} = -4.0
  \end{split}
\end{gather*}
\\よって$a = -4.0$である。
\\また、両辺に$(z-1)$をかけると、
\begin{equation*}
  (z-1)Y(z) = \frac{(z-1)}{(z-0.8)}a + b
\end{equation*}
この時、$z → 1$において、
\begin{equation*}
  (z-1)Y(z) = 0 + b
\end{equation*}
である。よって、
\begin{gather*}
  \begin{split}
    b &= \lim_{z → 1}(z-1)Y(z) \\
    b &= \frac{z}{z-0.8} \\
      &= \frac{1}{0.2} = 5.0
  \end{split}
\end{gather*}
\\よって$b = 5.0$である。
この時、$Y(z)$は、
\begin{equation*}
  \begin{split}
    Y(z)
    &= -4.0\frac{1}{z-0.8} + 5.0 \frac{1}{z-1}
  \end{split}
\end{equation*}
インパルス応答であるため、逆z変換は、
\begin{equation*}
  Z^{-1}\left[\frac{1}{z-\alpha}\right] =
  \begin{cases*}
    \alpha^{k-1} & (k \geqq 1) \\
    0 & (k < 1)
  \end{cases*}
\end{equation*}
\\となる。この式を$Y(z)$に当てはめて逆Z変換を行うと、
\begin{equation*}
  y_{k}=
  \begin{cases*}
    (-4.0) ・ 0.8^{k-1} + 5.0 & (k \geqq 1) \\
    0 & (k < 1)
  \end{cases*}
\end{equation*}
\\となる
\section*{問4}
差分方程式
\begin{equation*}
  y_{k+1} = 0.9y_{k} - 0.2y_{k-1} + u_{k}
\end{equation*}
\\の伝達関数表現を求めよ。
z変換より、
\begin{gather*}
  zY(z) = 0.9Y(z) - 0.2z^{-1}Y(z) \\
  U(z) = zY(z) - 0.9Y(z) + 0.2z^{-1}Y(z)
\end{gather*}
\\伝達関数を$G(z)$とすると、
\begin{gather*}
  Y(z) = G(z)U(z) \\
  U(z) = (z - 0.9 + 0.2z^{-1})Y(z) \\
  Y(z) = \frac{1}{(z - 0.9 + 0.2z^{-1})}U(z) \\
  \begin{split}
    G(z)
    &= \frac{1}{a-0.9+0.2z^{-1}} \\
    &= \frac{z}{z^{2}-0.9z+0.2}
  \end{split}
\end{gather*}
  よって、伝達関数表現は、
  \begin{equation*}
    G(z) = \frac{z}{z^{2}-0.9z+0.2}
  \end{equation*}
  である。
\section*{問5}
差分方程式
\begin{equation*}
  y_{k+1} = 1.1y_{k} - 0.3y_{k-1} + u_{k}
\end{equation*}
\\にステップ入力
\begin{equation*}
  u_{k}=
  \begin{cases*}
    1 & k \geq 0 \\
    0 & k < 0
  \end{cases*}
\end{equation*}
を加えたときの応答$y_{k}$を求めよ。

差分方程式をz変換する。
\begin{equation*}
  zY(z) = 1.1Y(z) - 0.3z^{-1}Y(z) + U(z)
\end{equation*}
\\移項すると、
\begin{gather*}
  U(z) = zY(z) - 1.1Y(z) + 0.3z^{-1}Y(z) \\
  U(z) = (z -1.1 + 0.3z^{-1})Y(z) \\
  Y(z) = \frac{z}{z^{2} -1.1z + 0.3}U(z)
\end{gather*}
$U(z)$を求める。
\begin{gather*}
  \begin{split}
    U(z) = Z[u_{k}]
    &= \sum_{i = -\infty}^{\infty}u_{i}z^{-i} \\
    &= \sum_{i = 0}^{\infty}1 ・ z^{-i} \\
    &= 1 + z^{-1} + z^{-2} \dotsb \\
    &= \frac{1}{1-z^{-1}} \ (|z^{-1} < 1|) \\
    &= \frac{z}{z-1}
  \end{split}
\end{gather*}
\\ここから、$U(z)$を代入してY(z)を部分分数分解する。
\begin{gather*}
  \begin{split}
    Y(z) 
    &= \frac{z}{z^{2}-1.1z+0.3}・\frac{z}{z-1} \\
    &= \frac{z^{2}}{(z-0.6)(z-0.5)(z-1)} \\
    &= \frac{a}{(z-0.6)}+\frac{b}{(z-0.5)}+\frac{c}{(z-1)}\\
  \end{split}
\end{gather*}
$Y(z)を通分して、係数比較を行う。$
\begin{equation*}
  Y(z) = \frac{1}{(z-0.6)(z-0.5)(z-1)}+
  ((a+b+c)z^{2}+(-1.5a-1.6b-1.1c)z+
  (0.5)a+0.6b+0.3c)
\end{equation*}
\\係数比較すると、
\begin{equation*}
  \begin{pmatrix}
    A = -9 \\
    B = 5 \\
    C = 5 \\
  \end{pmatrix}
\end{equation*}
\\よって、
\begin{equation*}
  Y(z) = -9・\frac{1}{z-0.6} + 5・\frac{1}{z-0.5} + 5・\frac{1}{z-1}
\end{equation*}
\\入力信号がステップ信号であるため、逆z変換の式は、
\begin{equation*}
  Z^{-1}\left[\frac{1}{z-\alpha}\right] =
  \begin{cases*}
    \alpha^{k-1} & (k \geqq 1) \\
    0 & (k < 1)
  \end{cases*}
\end{equation*}
\\逆z変換をY(z)に対して適用すると、
\begin{gather*}
  \begin{split}
    y_{k} 
    &= Z^{-1}[Y(z)] \\
    &=
    \begin{cases*}
      -9・0.6^{k-1} + 5・0.5^{k-1} + 5 & (k \geqq 1) \\
      0 & (k < 0)
    \end{cases*}
  \end{split}
\end{gather*}
\section*{問6}
伝達関数$H(z)$が$H(e^{j\omega}) = 3e^{-\frac{\pi}{4}j}$を満たすとすると、この
伝達関数に$u_{k}=\sin\omega k$を入力したときの定常応答を求めろ。


$H(e^{j\omega})$を極座標表示すると、
\begin{equation*}
  H(e^{j\omega}) = Ge^{j\phi}\\
\end{equation*}
\\この時、
\begin{equation*}
  H(e^{j^\omega}) = 3e^{-\frac{\pi}{4}j}
\end{equation*}
であるため、
\begin{equation*}
  \begin{pmatrix}
    G = |H(e^{j\omega k})| = 3 \\
    \phi = \angle H(e^{j \omega k}) = -\frac{\pi}{4}
  \end{pmatrix}
\end{equation*}
  \\以上より$u_{k} = \sin\omega k$を入力したときの定常応答は、
  \begin{equation*}
    y_{k} = 3 \sin (\omega k - \frac{\pi}{4})
  \end{equation*}
\end{document}