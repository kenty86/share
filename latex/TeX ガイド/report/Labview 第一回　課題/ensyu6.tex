\documentclass[a4paper]{ltjsarticle}
\begin{document}
\title{演習6 解析学Ⅱ}
\author{2022351033 関川謙人}
\maketitle
\section{Q1}

$f(x,y)=(x^{2}+y^{2})e^{x-y}$\label{f}

1段階微分すると、

$f_{x}(x,y)=(2x+x^{2}+y^{2})e^{x-y}$\label{fx}

$f_{y}(x,y)=(2y-x^{2}+y^{2})e^{x-y}$\label{fy}

この時、$f_{x}=0,f{y}=0$とする。
これを解くと、$(x,y)=(0,0),(1,1),(2,2)$の解が得られ、これが極値点の
候補となる。

また、\ref{fx},\ref{fy}を2段階微分すると、

$A=f_{xx}=(4x+x^{2}+y^{2}+2)e^{x-y}$\label{A}

$B=f_{xy}=(-x^{2}-2x+2y+y^{2})$\label{B}

$C=f_{yy}=(x^{2}-2y+2)e^{x-y}$\label{C}

$D=B^{2}-AC$\label{D}

\ref{D}に先ほど求めた解を代入すると、

$(0,0)$の場合

$A=0,B=0,C=0,D=0$であるため、$D=0$より、これだけでは判定不能である。

$(1,1)$の場合

$A=8,B=-2,C=1,D=-4$であるため、$D<0$より、この値は極値ではない。

$(2,2)$の場合、

$A=18,B=-8,C=2,D=28$であり、$D>0$、また、$A>0$であるため、\ref{f}は
$(x,y)=(2,2)$において極大値$8$をとる。

%%%%%%%%%%%%%%%%%
\section{Q2}
関係式を$f(x)=(x^{2}-xy+y^{2}-3)$と定義する。
$f(x)$の両辺を$x$で微分すると、

$f_{x}(x,y)=2x-y$

となる。

また、陰関数が極値をとるとき、

$f_{x}(x,y)=0,f(x,y)=0$
となるため、これらの式をとくと、極値をとるときの
$(a,b)$の値の候補は、$(a,b)=(0,0),(-1,-2),(1,2)$

となる。このとき、
$f_{y}(x,y)=-x+2y$

$f_{xx}(x,y)=-1$

また、

$\frac{f_{xx}(x,y)}{f_{y}(x,y)}=
\frac{2}{x-2y} $

である。

(-1,-2)の時、

$\frac{f_{xx}(x,y)}{f_{y}(x,y)}=-1<0$

なので、


$f(x)$は$x=(-1,-2)$で極小値$0$をとる。

(1,2)の時、
$\frac{f_{xx}(x,y)}{f_{y}(x,y)}=1<0$なので、
$f(x)$は$x=(1,2)$で極大値$0$をとる。

\end{document}