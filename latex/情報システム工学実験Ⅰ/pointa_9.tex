\documentclass[a4paper,titlepage,11pt]{ltjsarticle}
\usepackage{graphicx}
\usepackage{color}
\usepackage{amssymb}
\usepackage{here}
\usepackage{listings}
%プログラム挿入の設定
\lstset{
	%プログラム言語(複数の言語に対応,C,C++も可)
 	language = C,
 	%背景色と透過度
 	backgroundcolor={\color[gray]{.90}},
 	%枠外に行った時の自動改行
 	breaklines = true,
 	%自動開業後のインデント量(デフォルトでは20[pt])	
 	breakindent = 10pt,
 	%標準の書体E
 	basicstyle = \ttfamily\scriptsize,
 	%basicstyle = {\small}
 	%コメントの書体
 	commentstyle = {\itshape \color[cmyk]{1,0.4,1,0}},
 	%関数名等の色の設定
 	classoffset = 0,
 	%キーワード(int, ifなど)の書体
 	keywordstyle = {\bfseries \color[cmyk]{0,1,0,0}},
 	%""で囲まれたなどの"文字"の書体
 	stringstyle = {\ttfamily \color[rgb]{0,0,1}},
 	%枠 "t"は上に線を記載, "T"は上に二重線を記載
	%他オプション:leftline,topline,bottomline,lines,single,shadowbox
 	frame = TBrl,
 	%frameまでの間隔(行番号とプログラムの間)
 	framesep = 5pt,
 	%行番号の位置
 	numbers = left,
	%行番号の間隔
 	stepnumber = 1,
	%右マージン
 	%xrightmargin=0zw,
 	%左マージン
	%xleftmargin=3zw,
	%行番号の書体
 	numberstyle = \tiny,
	%タブの大きさ
 	tabsize = 4,
 	%キャプションの場所("tb"ならば上下両方に記載)
 	captionpos = t
}
\begin{document}

\begin{lstlisting}[caption=enshukadai9]
	void dijkstra(struct element *a[], int n, int s)
	{
		int i, u, v, sizeofT;
		struct element *l;
	
		for (i = 1; i <= n; i++){
			T[i] = 1;
			D[i] = INFTY; //2n回分初期化
		}
		T[s] = 0;
		D[s] = 0;
		for (l = a[s]->next; l != NULL; l = l->next)
			D[l->vertex] = l->weight;		//nlog(size)回計算
		sizeofT = n-1;
		while (sizeofT > 0){
			for (v = 1; v <= n; v++) 
				if (T[v] == 1)
		printf("v[%d], ", v);
				else
		printf("      ");
			for (v = 2; v <= n; v++) 
				printf("D[%d]=%d, ", v, D[v]);
			printf("\n");
			u = findmin(T, n);  //n回探索をする。
			T[u] = 0;          
			sizeofT--;
			for (v = 1; v <= n; v++)
				if (T[v] == 1)
		D[v] = min(D[v], D[u]+w[u][v]);//n回演算する。
		//2n+n+n=2nよって時間計算量O(2n)
		}
	}
\end{lstlisting}
\end{document}